% This file was converted to LaTeX by Writer2LaTeX ver. 1.4
% see http://writer2latex.sourceforge.net for more info
\documentclass[man,a4paper]{apa6}
\usepackage[ascii]{inputenc}
\usepackage[T1]{fontenc}
\usepackage[english]{babel}
\usepackage{amsmath}
\usepackage{amssymb,amsfonts,textcomp}
\usepackage{array}
\usepackage{supertabular}
\usepackage{hhline}
\usepackage{caption}
\usepackage{graphicx}
\usepackage[]{hyperref}
\hypersetup{
    bookmarksnumbered=true,     
    bookmarksopen=true,         
    bookmarksopenlevel=1,       
    colorlinks=true,            
    allcolors=blue,
    pdfstartview=Fit,           
    pdfpagemode=UseOutlines,    % this is the option you were lookin for
    pdfpagelayout=TwoPageRight
}

%\usepackage{setspace}
%\doublespace
\makeatletter
\newcommand\arraybslash{\let\\\@arraycr}
\makeatother
\setlength\tabcolsep{1mm}
\renewcommand\arraystretch{1.3}
\newcounter{Figure}
\renewcommand\theFigure{\arabic{Figure}}
% (suggestion) < 13 (key)words
\title{Attentional bias modification of rumination towards a self-relevant goal: A case-series}
\shorttitle{Attentional bias modification of rumination towards a self-relevant goal}
\author{Paul Sharpe}
\affiliation{University of Exeter}
\leftheader{Attentional bias modification of rumination towards a self-relevant goal: A case-series}
% use keywords
\abstract{300 words max}
% body: 8000 words max (excluding footnotes, tables, figures, references)
% appendices: 4000 words max (~250 words/page i.e. 16 pages)
\begin{document}
% FIXME: Include word count of main text on front page (see latexcount.pl)
\maketitle
\pdfbookmark{\contentsname}{toc}
\tableofcontents
\clearpage\setcounter{page}{1}


Mixed anxiety and depression is the most common mood disorder in Britain
(Haliwell, Main, \& Richardson, 2007).  Its predicted growth (World
Health Organisation, 2001) makes the associated negative individual and
socio-economic consequences a global concern.  In depression, relapse
is common and its likelihood increases steeply with each depressive
episode (Haliwell et al., 2007).  Anti-depressant medication (ADM)
and psychological therapies for depression such as CBT and MBCT can be
effective, but patients often dislike the side effects of ADM and there
are high costs associated with delivering psychological therapies.
Consequently, waiting times in Britain currently average six to
nine months, and can be as long as two years (Haliwell et al., 2007).
There is a need for cheap but effective interventions which target the
cognitive mechanisms underlying depression and anxiety.

Rumination is a thought process consistently linked to both the onset and maintenance of depression (Watkins \& Nolen-Hoeksema, 2014). \ In their review of the rumination literature, Smith and Alloy (2009) note that rumination as a construct is robust but that ``there is no unified definition [...] or standard way of measuring it'' (Smith \& Alloy, 2009 p. 117). \ \ To some extent, worrying about the future has been associated with anxiety and is distinguishable from brooding over the past which is more associated with depression (Papageorgiou \& Wells, 2004). \ However, other accounts (Watkins, Moulds, \& Mackintosh, 2005) suggest that both types of rumination may share the key qualities of an inability to control intrusive, repetitive thoughts in situations where they are not constructive, but are nevertheless perceived to be (Watkins, 2008). \ Rumination predicts symptoms of both anxiety and depression (Nolen-Hoeksema, 2000) and the conditions are frequently comorbid (Krusche, Cyhlarova, \& Williams, 2013). \ Therefore, `repetitive negative thought' (RNT) may be a more useful label which captures these similarities whilst distinguishing RNT from the constructive role played by repetitive thought in problem solving (Watkins, 2008).

Two accounts of rumination have been particularly influential. \ Response sStyles tTheory (Nolen-Hoeksema, 1991) defines rumination as a particular response to depressed mood where a person tends to focus on ``causes, consequences and symptoms of one's negative affect'' (Smith \& Alloy, 2009 p. 117) thereby prolonging low mood and depressive episodes. \ Control Theory (Martin \& Tesser, 1996) is broader in that it explains rumination as a thought process which arises when progress towards a goal is perceived to be unsatisfactory, a state within which low mood frequently co-occurs. \ Watkins and Nolen-Hoeksema (2014) have sought to unify these theories by proposing that RNT may be a habit, initiated by goal discrepancies and subsequently cued by contingent contexts such as mood or physical location. \ They suggest that such automatic, habitual responses come to depend on context more than the content of particular unresolved goals, especially when contingent responses involve passive focus on abstractly construed, negative content. \ Koster, De Lissnyder, Derakshan, and De Raedt (2011) claim that cued repetitive thought per-se is a normal cognitive response to self-evaluation, but hypothesise that one driver of RNT, particularly brooding, is an impaired attentional ability to disengage from negative thoughts (Koster et al., 2011).

\ \ Although habitual RNT is resistant to change (Watkins \& Nolen-Hoeksema, 2014), it seems possible that it could be altered using counterconditioning techniques. \ Counterconditioning is one of the basic mechanisms which cognitive bias modification (CBM) employs in the form of repetitive computer tasks to attenuate attentional, interpretive and memory biases associated with negative emotional states (Hertel \& Mathews, 2011). \ Randomised controlled trials indicate that CBM may be an effective treatment for anxiety (Hakamata et al., 2010) and depression (Yang, Ding, Dai, Peng, \& Zhang, 2014). \ Attentional bias modification (ABM) is a common form of CBM training in which participants use a computer task to repeatedly attend towards or away from a particular stimulus such as negatively valenced words or pictures (see MacLeod and Clarke, 2015 for a recent review). \ This type of training resembles the counterconditioning mechanism which Watkins and Nolen-Hoeksema (2014) suggest may reduce RNT. \ A recent randomised controlled trial of ABM on individuals with depressive symptoms (Yang et al., 2014) met the independent, but essential criteria of a successful ABM intervention stressed by MacLeod and Clarke (2015). \ First, a post-training change in attentional bias was demonstrated. \ Second, and notable in the context of the habit-goal framework, this effect on reducing depressive symptoms was mediated by rumination (Yang et al., 2014). \ Yang et al. (2014) used laboratory-based ABM training using sad word stimuli. \ The proposed study aims to extend these findings by testing whether ABM reduces RNT and elevates mood in non-depressed, high ruminators exposed to \ abstract, negative, ideographic cue words associated with an incomplete self-relevant goal. \ This is a novel and more direct test of RNT within the habit-goal framework (Watkins \& Nolen-Hoeksema, 2014), as this arrangement should create a context likely to trigger RNT within which the counterconditioning effects of ABM training can be measured.

Reduced attentional bias, state rumination, state anxiety and low mood scores are predicted after ABM training (phase B) in comparison with baseline (phase A) measures. \ Initial measures of trait rumination, trait anxiety and depression are also predicted to reduce after ABM training. \ Positive findings would add weight to the impaired disengagement hypothesis (Koster et al., 2011) and the habit-goal framework (Watkins \& Nolen-Hoeksema, 2014) and would be proof of principle for a potential clinical application of ABM to change RNT.

\section{Method}
\subsection{Design}
A single-case series design was chosen as this is a small-scale, proof-of-principle study with limited funds for recruiting participant numbers required for a group study. \ More specifically, an AB randomised sequential replication design (Onghena \& Edgington, 2005) will be used to allow a comparison of each participant's baseline (phase A) scores against their post-training (phase B) scores and a meta-analysis across all participants.

\subsection{Participants}
The University of Exeter Research Participation System (SONA) and on-campus posters will be used to recruit 10 student participants who will be invited to evaluate a computer-based experimental task which may improve how they relate to an unresolved problem. \ Participants will receive course credits and/or be entered into a draw for a {\pounds}20 prize. \ Inclusion criteria will require no concurrent psychotherapy or psychotropic medication and a PHQ-8 score {\textless} 10 (Kroenke et al., 2009). \ To ensure enough participants are recruited, a liberal inclusion criterion of Ruminative Responses Scale (RRS) ${\geq}$ 50 (Treynor, Gonzalez, \& Nolen-Hoeksema, 2003) will be used initially. \ If recruitment responses are high then this threshold may be increased to recruit participants with the highest possible RRS scores.

\subsection{Measures}
Anxiety and depression will both be measured due to their close association, and uncertainty whether the procedure will affect brooding, worry or both. \ Measurements will be taken pre and post-ABM training for trait rumination using the RRS (Treynor et al., 2003), and trait anxiety using the STAI (Spielberger, 1983). \ Scales will relate to the previous 4 weeks of participants' lives. \ Depression will be measured pre and post-ABM training using the PHQ-9, which has good diagnostic performance for screening for Major Depressive Disorder (Manea, Gilbody, \& McMillan, 2015). \ Primary measures will be taken daily during the case series for attentional bias using the dot-probe (MacLeod, Mathews, \& Tata, 1986), state rumination using the Goal Rumination Scale (GRS) (Schultheiss, Jones, Davis, \& Kley, 2008), and state anxiety using the STAI-T (Spielberger, 1983). \ Secondary daily measures of mood will be taken using the I-PANAS-SF (Mackinnon et al., 1999) supplemented with 2 items 'Sad' and 'Anxious'. \ The GRS and STAI-T and I-PANAS-SF will cover the last 24 hours of participants' lives. \ Age and gender demographics will also be recorded.

\subsection{Materials}
Two sets of visual word stimuli will be used. \ The 18 words chosen by the participant in relation to their incomplete goal will be referred to as the 'G set'. \ The 'P set' will be common to all participants and comprise 167 words selected from a pool of abstract, sad words following the procedure in MacLeod, Rutherford, Campbell, Ebsworthy, \& Holker (2002). \ Each word from the G and P sets will be paired with a corresponding neutral word, matched for letter length, frequency of usage (van Heuven, Mandera, Keuleers, \& Brysbaert, 2014), valence, arousal, imagery, familiarity and relevance to sadness (Yang et al., 2014). \ In contrast with the abstract words, the neutral pair words will be concrete. \ The rationale for this is that concreteness training has been shown to reduce depressive symptoms and state rumination (Watkins \& Moberly, 2009), therefore ABM training which directs attention towards neutral word locations may be enhanced if the negative words in the pairs are abstract but the neutral words are concrete. \ The total of 185 word pairs allows for 1 practice and 36 experimental dot-probe test sessions even when the 18 G set words fully overlap with P set words. \ Dot-probe and ABM tasks will be developed using OpenSesame (Math\^ot, Schreij, \& Theeuwes, 2011) and measurement instruments administered online using Lime Survey.

\subsection{Attentional Bias Modification (ABM)}
\subsubsection{Dot-probe task}
The dot-probe task will be based on Yang et al. (2014), and consist of 96 trials (see Figure 1).



\begin{center}
\begin{minipage}{20.99cm}
  [Warning: Image ignored] % Unhandled or unsupported graphics:
%\includegraphics[width=16.459cm,height=7.103cm]{proposal-img001.svm}
 

Figure \stepcounter{Figure}{\theFigure}: Example dot-probe sequence where the negative word appears at the upper location (Trials 1 \& 2) and the probe appears at the neutral location (Trial 1) and negative location (Trial 2).
\end{minipage}
\end{center}
Each trial will begin with an 8mm x 8mm white fixation cross centred on a black screen. \ After 500-ms, this will be replaced by a word pair from a subset of 5 word pairs, 4 from the P set and 1 from the G set. \ One word from the pair will appear above the fixation cross location and one below. \ Word pairs will be 50mm high with 30mm separating the words. \ Negative words will occur with equal frequency at the upper or lower positions. \ After 2000 ms, the word pair will disappear and one 3mm diameter dot or two 3mm diameter dots with a 2mm centre-to-centre distance will appear at one of the previous word locations. \ Dot-probes will appear with equal frequency at negative and neutral locations. \ Participants will respond to the single dot by pressing the left mouse button and the two dots by pressing the right mouse button. \ After responding, a random inter-trial interval (ITI) between 100-500 ms with a blank screen will precede the next trial. \ Participants will be asked to respond as quickly and accurately as possible. \ Each dot-probe block will take approximately 5 minutes.

\subsubsection{ABM task}
The ABM task will be identical to the dot-probe task but, to train attentional allocation away from the negative words, will consist of 192 trials using word pairs from both G and P sets, with probes which always appear at the neutral word location. \ Each ABM block will take approximately 15 minutes.

\subsubsection{Attentional Bias Assessment}
Inaccurate trials or those with response times exceeding 3 standard deviations beyond the mean will be excluded. \ Attentional bias scores will be calculated from the remaining response times using the equation

score = [(NuPl + NlPu) = (NuPu + NlPl)]/2 \ \ \ \ (1)

where N = Negative word, P = Probe, u = upper, t = lower (Bradley et al., 1997).

\subsection{Procedure}
An introductory session will take place at the University of Exeter after the participant gives informed, written consent to take part in the study. \ A self-relevant, unresolved goal will be chosen by asking participants to think of {}``an ongoing and unresolved concern that [has] repeatedly come into their mind and caused them to feel negative or stressed during the previous week'' (Roberts, Watkins, and Wills, 2013, p. 451). \ Examples of appropriate problems will be provided. \ Participants will then be asked to generate a list of at least 18 words describing the personal ``causes, meanings and implications'' (Watkins, Baeyens, \& Read, 2009, p. 55) of their unresolved goal. \ The wording will encourage abstract construal, which has been associated with RNT (Watkins et al., 2009). \ The experimenter will leave the room after the participant begins writing their word list. \ After 15 minutes the experimenter will return and guide the participant to create the 'G set' by ranking the top 18 words in decreasing order of negative affect which they evoke in relation to their goal.

Participants will complete measures of trait rumination (RRS), trait anxiety (STAI) and depression (PHQ-9) using the Lime Survey website, before being introduced to the experimental task. \ They will be instructed to sit about 60 cm from the computer screen and to allocate sufficient time in this, and future sessions, to complete the session without interruption (a maximum of about 20 minutes for training blocks and self-report measures). \ Participants will be encouraged to carry out the task at time of day which will be consistently convenient throughout the study. \ A training dot-probe test session will be delivered to record the initial attentional bias measurement. \ This practice session will include 5 P set words and no G set words. \ On completion, the experimenter will answer any questions regarding the task, explain the schedule for subsequent days and remind the participant of their 1 in 10 chance of winning a {\pounds}20 prize. \ After the participant leaves, the experimenter will match words to create the G set. \ Participants will carry out all experimental sessions on a computer at their home. \ The total study length for each participant will be 36 days.

\subsubsection{Day 1}
The experimenter will telephone participants on the first day of the experiment to help them install any software required to run the dot-probe and ABM tasks. \ Participants will follow emailed instructions guiding them to complete state rumination (GRS), state anxiety (STAI-T) and mood (I-PANAS-SF) measurements followed by the first phase A dot-probe test block. \ The experimenter will call the participant after this first session to answer any questions regarding the task.

\subsubsection{Days 2 to 36}
Participants will be emailed daily, instructing them to first complete the measurement instruments followed by instructions on how to run the computer task which matches the next phase in their randomisation schedule. \ This will be a test block for phases A and B, followed by an ABM training block in phase B only (see Figure 1). \ The experimenter will telephone participants twice in week 1 to ensure their well-being and motivation. \ Thereafter they will be telephoned weekly and the day after any missed session. \ In the final week, the experimenter and participant will agree a time for the debriefing session on the day following the last experimental session.

\subsubsection{Day 37}
\ The final session will take place at the University of Exeter. \ Participants will complete trait (RRS) rumination, trait anxiety (STAI) and depression (PHQ-9) measures and will be invited to write down their current feelings towards their personal goal and overall experience of the training before being thanked for their participation, remunerated and debriefed.

\subsubsection[Stimulus allocation]{Stimulus allocation}
\ \ See et al. (2009) note that the most desirable change in attentional bias is in relation to emotional tone in general rather than to any specific stimuli. \ To maximise both the measurement and modification of attention to emotional tone in general, each assessment block will only contain stimulus materials not previously exposed in any preceding training block (See, MacLeod, \& Bridle, 2009). \ Table 1 provides an example of how subsets of the P and G sets will be introduced as the experiment progresses. \ Dot-probe blocks will always consist of 5 word pairs; the next 4 pairs from the P set and 1 pair from the G set. \ The G pair will change to the next highest rated pair every 2 blocks. \ To maximise stimulus variety during ABM, these blocks will include word pairs from all preceding dot-probe blocks. \ Thus, the number of stimulus word pairs will increase with each ABM block. \ To emphasise ideographic stimuli, G pairs will appear twice as frequently as P pairs. \ Each word in the pair will appear with equal frequency at upper and lower screen locations. \ Where necessary, stimuli will be randomly excluded from a stimulus subset until the total exposures for the block is exactly divisible by the stimulus subset.

\captionof{table}[]{}
Example stimulus allocation for a randomisation pattern which transitions from phase A (dot-probe test) to phase B (ABM training) at session 12.

\begin{flushleft}
\tablefirsthead{}
\tablehead{}
\tabletail{}
\tablelasttail{}
\begin{supertabular}{m{1.599cm}m{3.001cm}m{1.4929999cm}m{1.4139999cm}m{0.38200003cm}m{1.52cm}m{1.599cm}m{1.599cm}m{0.48800004cm}m{1.8109999cm}}
\hline
\multicolumn{2}{m{4.8cm}}{\centering Day} &
\centering 1 &
\centering 2 &
\centering ... &
\centering 11 &
\centering 12 &
\centering 13 &
\centering ... &
\centering\arraybslash 36\\\hline
\multicolumn{2}{m{4.8cm}}{\centering Phase} &
\centering A &
\centering A &
\centering A &
\centering A &
\centering B &
\centering B &
\centering B &
\centering\arraybslash B\\\hline
Dot-probe &
Stimulus subset &
\centering P1G1 &
\centering P2G1 &
\centering ... &
\centering P11G6 &
\centering P12G6 &
\centering P13G7 &
\centering ... &
\centering\arraybslash P36G18\\\hline
 &
Exposures/P pairs &
\centering 48/4 = 12 &
\centering 48/4 = 12 &
\centering ... &
\centering 48/4 = 12 &
\centering 48/4 = 12 &
\centering 48/4 = 12 &
\centering ... &
\centering\arraybslash 48/4 = 12\\\hhline{~---------}
 &
Exposures/G pairs &
\centering 48/1 = 48 &
\centering 48/1 = 48 &
\centering ... &
\centering 48/1 = 48 &
\centering 48/1 = 48 &
\centering 48/1 = 48 &
\centering ... &
\centering\arraybslash 48/1 = 48\\\hhline{~---------}
ABM &
Stimulus subset &
~
 &
~
 &
~
 &
~
 &
\centering P1-12G1-6 &
\centering P1-13G1-7 &
\centering ... &
\centering\arraybslash P1-36G1-18\\\hline
 &
Exposures/P pairs &
~
 &
~
 &
~
 &
~
 &
\centering 96/48 = 2 &
\centering 96/52ab &
\centering ... &
\centering\arraybslash 96/144b\\\hhline{~---------}
 &
Exposures/G pairs &
~
 &
~
 &
~
 &
~
 &
\centering 96/12 = 8 &
\centering 96/12 = 8 &
\centering ... &
\centering\arraybslash 96/18a\\\hhline{~---------}
\end{supertabular}
\end{flushleft}
a Where necessary, stimuli will be randomly excluded until exposures are exactly divisible by total stimuli.

b Where total stimuli * 2 {\textgreater} exposures, stimuli will be randomly excluded until total stimuli * 2 = exposures.

\section{Analyses}

\subsection{proposed}

\ \ Randomisation procedures will be used to allow nonparametric inferential statistics to be calculated for each participant (Bult\'e \& Onghena, 2008). \ With a minimum of 8 ABM training sessions and 36 measurements in total, there are 21 possible AB transition points, which allows significance testing above the .05 level for each case (Onghena \& Edgington, 2005). \ To avoid having to discard complete cases where a small number sessions are not completed, missing values for these sessions will be replaced by values carried forward the the previous day's measurements.

Visual analyses of each case series will be made using graphs generated by the R package 'SCVA' (Bult\'e \& Onghena, 2012). \ Inferential statistics comparing \ A and B phases for measures of attentional bias, trait rumination, trait anxiety and mood within each case series will be calculated using nonparametric randomisation tests provided by the R package 'SCRT' (Bult\'e \& Onghena, 2008). \ Effect sizes and a meta-analysis across all participants will be calculated using the R package 'SCMA' (Bult\'e \& Onghena, 2013). \ Paired samples t-tests will be used to compare pre and post ABM scores for trait rumination, trait anxiety and depression scores, although these will be mostly descriptive as they will be underpowered.

\subsection{planned}
\begin{APAenumerate}
     \item Multiple comparison corrections
\end{APAenumerate}

\section{Ethical Considerations}
Because recruitment measures will be submitted remotely, depression will be measured using the PHQ-8 which omits one item measuring suicidal ideation. \ Anyone scoring ${\geq}$ 10 on the PHQ-8 (Kroenke et al., 2009) will be signposted towards resources helpful for people who may be suffering from depression. \ The PI completed risk training at the Mood Disorders Centre in January 2015 and the process for implementing the approved protocol will be rehearsed with the PI's supervisor in advance. \ The protocol will be triggered by participants scoring ${\geq}$ 10 on their initial PHQ-9 and final PHQ-9. \ The potential risk of harm arising from increasing rumination in high trait ruminators, along with the burden of a long testing period, will be continuously monitored in the regular contact that the PI makes with participants. \ Any concerns will be raised with the PI's supervisor. \ Participants will not be identifiable from their electronic or written data. \ The former will be stored on a secure \ digital filing system and the latter in a locked filing cabinet.


\bigskip

2981 words


\bigskip

\clearpage\section{References}
Bradley, B. P., Mogg, K., \& Lee, S. C. (1997). Attentional biases for negative information in induced and naturally occurring dysphoria. Behaviour Research and Therapy, 35(10), 911--927. doi:10.1016/S0005-7967(97)00053-3

Bufton, S. (2003). The Lifeworld of the University Student: Habitus and Social Class. Journal of Phenomenological Psychology, 34(2), 207--234.

Bult\'e, I., \& Onghena, P. (2008). An R package for single-case randomization tests. Behavior Research Methods, 40(2), 467--478. doi:10.3758/BRM.40.2.467

Bult\'e, I., \& Onghena, P. (2012). When the Truth Hits You Between the Eyes A Software Tool for the Visual Analysis of Single-Case Experimental Data. Methodology-European Journal of Research Methods for the Behavioral and Social Sciences, 8(3), 104--114.

Bult\'e, I., \& Onghena, P. (2013). The Single-Case Data Analysis Package: Analysing Single-Case Experiments with R Software. Journal of Modern Applied Statistical Methods, 12(2). Retrieved from http://digitalcommons.wayne.edu/jmasm/vol12/iss2/28

Hakamata, Y., Lissek, S., Bar-Haim, Y., Britton, J., Fox, N., Leibenluft, E., {\dots} Pine, D. (2010). Attention Bias Modification Treatment: A Meta-Analysis Toward the Establishment of Novel Treatment for Anxiety. Biological Psychiatry, 68(11), 982--990.

Haliwell, E., Main, L., \& Richardson, C. (2007). The Fundamental Facts. Mental Health Foundation. Retrieved from http://www.mentalhealth.org.uk/content/assets/PDF/publications/fundamental\_facts\_2007.pdf?view=Standard

Hertel, P., \& Mathews, A. (2011). Cognitive Bias Modification: Past Perspectives, Current Findings, and Future Applications. Perspectives on Psychological Science, 6(6), 521--536.

Koster, E. H. W., De Lissnyder, E., Derakshan, N., \& De Raedt, R. (2011). Understanding depressive rumination from a cognitive science perspective: The impaired disengagement hypothesis. Clinical Psychology Review, 31(1), 138--145. doi:10.1016/j.cpr.2010.08.005

Kroenke, K., Strine, T. W., Spitzer, R. L., Williams, J. B. W., Berry, J. T., \& Mokdad, A. H. (2009). The PHQ-8 as a measure of current depression in the general population. Journal of Affective Disorders, 114(1--3), 163--173. doi:10.1016/j.jad.2008.06.026

Krusche, A., Cyhlarova, E., \& Williams, J. M. G. (2013). Mindfulness online: an evaluation of the feasibility of a web-based mindfulness course for stress, anxiety and depression. BMJ Open, 3(11), e003498. doi:10.1136/bmjopen-2013-003498

Mackinnon, A., Jorm, A. F., Christensen, H., Korten, A. E., Jacomb, P. A., \& Rodgers, B. (1999). A short form of the Positive and Negative Affect Schedule: evaluation of factorial validity and invariance across demographic variables in a community sample. Personality and Individual Differences, 27(3), 405--416. doi:10.1016/S0191-8869(98)00251-7

MacLeod, C., \& Clarke, P. J. F. (2015). The Attentional Bias Modification Approach to Anxiety Intervention. Clinical Psychological Science, 3(1), 58--78. doi:10.1177/2167702614560749

MacLeod, C., Mathews, A., \& Tata, P. (1986). Attentional bias in emotional disorders. Journal of Abnormal Psychology, 95(1), 15--20. doi:10.1037/0021-843X.95.1.15

MacLeod, C., Rutherford, E., Campbell, L., Ebsworthy, G., \& Holker, L. (2002). Selective attention and emotional vulnerability: Assessing the causal basis of their association through the experimental manipulation of attentional bias. Journal of Abnormal Psychology, 111(1), 107--123. doi:10.1037/0021-843X.111.1.107

Manea, L., Gilbody, S., \& McMillan, D. (2015). A diagnostic meta-analysis of the Patient Health Questionnaire-9 (PHQ-9) algorithm scoring method as a screen for depression. General Hospital Psychiatry, 37(1), 67--75. doi:10.1016/j.genhosppsych.2014.09.009

Martin, L. L., \& Tesser, A. (1996). Some ruminative thoughts. In Ruminative thoughts (pp. 1--47). Hillsdale, NJ, England: Lawrence Erlbaum Associates, Inc.

Math\^ot, S., Schreij, D., \& Theeuwes, J. (2011). OpenSesame: An open-source, graphical experiment builder for the social sciences. Behavior Research Methods, 44(2), 314--324. doi:10.3758/s13428-011-0168-7

Nolen-Hoeksema, S. (1991). Responses to depression and their effects on the duration of depressive episodes. Journal of Abnormal Psychology, 100(4), 569--582. doi:10.1037/0021-843X.100.4.569

Nolen-Hoeksema, S. (2000). The role of rumination in depressive disorders and mixed anxiety/depressive symptoms. Journal of Abnormal Psychology, 109(3), 504--511. doi:10.1037/0021-843X.109.3.504

Onghena, P., \& Edgington, E. S. (2005). Customization of Pain Treatments: Single-Case Design and Ana...[202F?]: The Clinical Journal of Pain. Clinical Journal of Pain, 21(1). Retrieved from http://journals.lww.com/clinicalpain/Fulltext/2005/01000/Customization\_of\_Pain\_Treatments\_\_Single\_Case.7.aspx

Papageorgiou, C., \& Wells, A. (2004). Depressive Rumination: Nature, Theory and Treatment. John Wiley \& Sons.

Roberts, H., Watkins, E., \& Wills, A. (2013). Cueing an unresolved personal goal causes persistent ruminative self-focus: An experimental evaluation of control theories of rumination. Journal of Behavior Therapy and Experimental Psychiatry, 44(4), 449--455.

Schultheiss, O. C., Jones, N. M., Davis, A. Q., \& Kley, C. (2008). The role of implicit motivation in hot and cold goal pursuit: Effects on goal progress, goal rumination, and emotional well-being. Journal of Research in Personality, 42(4), 971--987. doi:10.1016/j.jrp.2007.12.009

See, J., MacLeod, C., \& Bridle, R. (2009). The reduction of anxiety vulnerability through the modification of attentional bias: A real-world study using a home-based cognitive bias modification procedure. Journal of Abnormal Psychology, 118(1), 65--75. doi:10.1037/a0014377

Smith, J. M., \& Alloy, L. B. (2009). A roadmap to rumination: A review of the definition, assessment, and conceptualization of this multifaceted construct. Clinical Psychology Review, 29(2), 116--128. doi:10.1016/j.cpr.2008.10.003

Spielberger, C. D. (1983). Manual for the State-trait anxiety inventory (form Y) (``self-evaluation questionnaire''). Consulting Psychologists Press.

Treynor, W., Gonzalez, R., \& Nolen-Hoeksema, S. (2003). Rumination Reconsidered: A Psychometric Analysis. Cognitive Therapy and Research, 27(3), 247--259. doi:10.1023/A:1023910315561

Van Heuven, W. J. B., Mandera, P., Keuleers, E., \& Brysbaert, M. (2014). SUBTLEX-UK: A new and improved word frequency database for British English. The Quarterly Journal of Experimental Psychology, 67(6), 1176--1190. doi:10.1080/17470218.2013.850521

Watkins, E., Moulds, M., \& Mackintosh, B. (2005). Comparisons between rumination and worry in a non-clinical population. Behaviour Research and Therapy, 43(12), 1577--1585.

Watkins, E., \& Nolen-Hoeksema, S. (2014). A Habit-Goal Framework of Depressive Rumination. Journal of Abnormal Psychology, 123(1), 24--34.

Watkins, E. R. (2008). Constructive and unconstructive repetitive thought. Psychological Bulletin, 134(2), 163--206. doi:10.1037/0033-2909.134.2.163

Watkins, E. R., Baeyens, C. B., \& Read, R. (2009). Concreteness training reduces dysphoria: Proof-of-principle for repeated cognitive bias modification in depression. Journal Of Abnormal Psychology, 118, 55--64.

Watkins, E. R., \& Moberly, N. J. (2009). Concreteness training reduces dysphoria: A pilot proof-of-principle study. Behaviour Research and Therapy, 47, 48--53.

Yang, W., Ding, Z., Dai, T., Peng, F., \& Zhang, J. X. (2014). Attention Bias Modification training in individuals with depressive symptoms: A randomized controlled trial. Journal of Behavior Therapy and Experimental Psychiatry. doi:10.1016/j.jbtep.2014.08.005
\end{document}
% FIXME: Ethics appendix (email without identifying details)
