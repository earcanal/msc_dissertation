Thank you for taking part in this study which was designed to examine
whether daily attention training can reduce levels of rumination and/or
improve mood. Rumination is associated with both depression and anxiety
and is known to predict the length and severity of depressive episodes,
as well as the risk of relapse in the future. Researchers and clinicians
are investigating this type of task as a way to reduce the negative
affects of rumination to supplement existing treatments.

There is evidence to suggest that rumination is linked with a difficulty
disengaging from negative thoughts and that this tendency can be reduced
using computer-based training programmes. There is also evidence that
people tend to ruminate over self-relevant, unresolved goals. This is why
you were asked to take part in a 35-day study looking at the effects of
attention training on the day-to-day tendency to ruminate in response to
negative words in general, and negative words relating to an unresolved
goal of your own. We were also interested to see whether the training led
to improvements in daily mood and changes to levels of depression before
and after the training. Scores on each measure will be compared before
and after participants began the attention training, to see whether it
had the desired effect.  One of the computer tasks you carried out each
day measured your attentional orientation towards negative and neutral
words. In the second phase of the study, this was preceded by a task
designed to reduce your attentional focus on negative words.

Over the course of the study, you have accrued a payment of \pounds{X}. Please
contact the researcher to indicate which of the following methods you
would prefer

\begin{enumerate}
  \item Receive cash from researcher or supervisor at the University of Exeter
  \item Electronic payment into your bank account
\end{enumerate}

Now that you have completed/withdrawn from the study, the researcher will
delete the link between your participant number and personal details so
that all information provided by you will be stored anonymously. This
means it will not be possible to trace your scores on any measure back
to you personally. A consequence of this is that the researcher will
no longer be able to identify which scores belong to which participant,
so you will not be able to withdraw your data once this happens. Please
let the researcher know by \texttt{<date>} if you wish to withdraw your
data from the study.  After this date, your scores will be anonymised
and it will no longer be possible to withdraw.

You can remove the software and data associated with the computer task
by deleting the experiment folder which you created on your computer at
the start of the study. If you were using a Macintosh computer, you can
also uninstall the OpenSesame software.

The results of this study may be published within academic journals
and shared at conferences. All information will be anonymised so that
it will not be possible to identify any participants personally. If you
are interested in finding out the results of the study, please contact
the researcher via email after 31\textsuperscript{st} October 2015 to
request a summary.

If you have any questions about this study, please contact the researcher
via email (see below). Questions or concerns about this study can also be
addressed to the Chair of the Ethics Committee, School of Psychology,
University of Exeter. He can be contacted at the following email
address: \texttt{<email>}. You can also contact the research supervisor,
\texttt<XXX>, by emailing: \texttt{<email>}.

I hope you found the study interesting. I'm really grateful for your
participation and the significant contribution it has made to my research.
