
\setlength\tabcolsep{1mm}
\belowrulesep=0pt
\aboverulesep=0pt
\begin{sidewaystable}[!htbp] \centering
\begin{threeparttable}
{\footnotesize
  \begin{tabular}{@{} |l*{19}{|c}|l| @{}}
    \noalign{\hrule height 1.5pt}
    Pre & \multicolumn{2}{c|}{} & \multicolumn{8}{c|}{Phase A} & \multicolumn{9}{c|}{Phase B} & Post \\
    \hline
    RRS & \multicolumn{2}{c|}{} & \multicolumn{6}{c}{} & & ABM\tabfnm{b} N-words: & 1--6 & 1--12 & 1--18 & 1--24 &
    1--30 & 1--36 & 1--42 & 1--48 & 1--54 & RRS \\
    \cmidrule{2-20}
    PSWQ & dot-probe\tabfnm{a} & N-words & 1--6 & 7--12 & 13--18 &
    19--24 & 25--30 & 31--36 & 37--42 & 43--48 & 7--12 & 13--18 & 19--24 & 25-30
    & 31--36 & 37--42 & 43--48 & 49--54 & 55--60 & PSWQ \\
    \cmidrule{3-20}
    PHQ-9 & & I-words & 1--6 & 1--6 & 1--6 & 1--6 & 1--6 & 1--6 & 1--6 & 1--6 & 1--6
    & 1--6 & 1--6 & 1--6 & 1--6 & 1--6 & 1--6 & 1--6 & 1--6 & PHQ-9 \\
    \cmidrule{2-20}
     GAD-7 & & Session & 1 & 2 & 3 & 4 & 5 & 6 & 7 & 8 & 9--11 & 12--14 & 15--18 &
    19--21 & 21--23 & 24--26 & 27--29 & 30--32 & 33--35 & GAD-7 \\
    \hline
    \end{tabular}
} % footnotesize
    \caption{Example randomisation schedule showing stimulus allocation
    and block order where phase A = 8 sessions and phase B = 27 sessions.
    In phase A, dot-probe blocks consisted of 12 word pairs; the 6 I-word
pairs and 6 N-word pairs. Session 1 used pairs 1--6 from the 66 N-word
set, session 2 used pairs 7--12, and so on. Thus, bias associated with
the participant's ideographic words was measured at every session,
whereas the more general negative valence associated with N-words
was distributed across measurement times. In phase B, dot-probe
measurement was preceded by ABM blocks containing only N-words. To guarantee
enough stimuli when randomisation resulted in a long phase B, the same 6
N-words were used for 3 consecutive ABM training sessions. N-words from
previous ABM blocks were included in subsequent ABM blocks. For example,
the first 3 ABM blocks used N-words 1--6 and the subsequent dot-probe
used N-words 7--12. The fourth ABM block used N-words 1--12 and the
subsequent dot-probe used N-words 13--18, and so on. A random draw was
used to select the 192 stimuli for each ABM session. As with phase A,
the 6 I-words were present in every dot-probe block.}
    \label{tab:stim}
\begin{tablenotes}[para,flushleft]
{\footnotesize
  \tabfnt{a}96 trials
  \tabfnt{b}192 trials
}
\end{tablenotes}
\end{threeparttable}
\end{sidewaystable}
