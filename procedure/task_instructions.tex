\subsection{Overview}

The computer task will initially take about 5 minutes to complete. Later
in the study, this will increase to about 15 minutes. You will need
to follow the instructions in an email you receive for each daily
'session'. These will remind you to complete the computer task and
answer a short set of questions using a web browser. Following the
instructions in the email will ensure that the data from the computer
task is matched to your answers for that session.  Remember to delete
the email at the end of the session to avoid accidentally repeating an
earlier session. You will only receive a reminder for the next session
once you have completed the current session.

Ideally you will complete all sessions on consecutive days. However, don't
worry too much if your schedule means you have to miss a session. The
researcher will contact you if this happens to see whether you want to
complete that session on the next day (extending your participation by
a day) or to skip that particular session. The emails you received will
repeat the information you need for the session you are about to complete.

\subsection{Hardware and software requirements}
\begin{itemize}
\item Computer running Windows, Macintosh (OS X 10.9 or later), or Linux operating system software
\end{itemize}

\subsection{Installing and running the computer task}

The researcher will provide you with specific instructions to install
and run the computer task on the operating system (Windows, Macintosh,
Linux) installed on your own computer. The software you will install
is a standard package for running psychological experiments. This has
been tested, and should not otherwise affect the running your computer
and can be removed at the end of the study. Whilst the researcher will
be available to provide technical help, installation and removal of
software is ultimately your responsibility. The researcher will analyse
your response accuracy daily, to ensure that the experiment is running
smoothly.

\subsection{How to complete the task}

Please try to observe the following when running the computer task

\begin{itemize}
\item Choose at a time when you are unlikely to be interrupted or are trying to do other things (turn off mobile phones,
televisions etc.)
\item Before running the task, close all other programs on your computer
\item Sit upright, approximately 60cm from the screen
\item Use the same computer to complete each session.
\end{itemize}

\subsection{Data Files}

Each day you will upload a data file generated by the computer
task. These are stored in the folder \texttt{experiment/data}
in the location where you originally extracted the computer
task. The form which asks you to upload your data file will
include the name of the file to upload. Each data file has the name
\texttt{p{\textless}participant{\textgreater}s{\textless}session{\textgreater}},
where \texttt{{\textless}participant{\textgreater}} is your participant
number and \texttt{{\textless}session{\textgreater}} is the session
number (1-35) you have just completed. For example, after completing
the task for session 25, participant 2 would upload the file
\texttt{experiment/data/p2s25}.

\subsection{Dealing with problems during the study}

I appreciate that there may be some early technical issues to overcome,
so if you have any problems with the instructions, do email or call me
as I should be able to help. Please use the same contact details if you
are concerned about any major worsening of your mood during the study.

\subsection{Finally...}

Once you have the computer task running at home, you've completed most
of the more difficult set-up tasks. After completing the first few
sessions you will become familiar with the tasks and should find a way
of fitting them into your daily routine without too much disruption. I
hope you enjoy completing the daily sessions, and want to thank you for
your commitment to the study.
