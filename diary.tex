\documentclass[british]{article}
\usepackage{babel}
\usepackage{csquotes}
\usepackage[backend=biber,date=short,maxcitenames=2,style=apa]{biblatex}
\DeclareLanguageMapping{british}{british-apa}

\usepackage{hyperref}
\hypersetup{pdftex, colorlinks=true, linkcolor=blue, citecolor=blue, filecolor=blue, urlcolor=blue, pdftitle=, pdfauthor=Paul Sharpe, pdfsubject=, pdfkeywords=}

\DeclareSourcemap{
  \maps[datatype=bibtex]{
    \map{
       \step[fieldsource=sortname]
       \step[fieldset=namea, origfieldval, final]
    }
  }
}
\DeclareLabelname{
  \field{namea}
  \field{shortauthor}
  \field{author}
  \field{shorteditor}
  \field{editor}
  \field{translator}
}
\addbibresource{diary.bib}

\title{}
\author{Paul Sharpe}
\date{2015-09-16}

\usepackage[margin=1in]{geometry}
\begin{document}

\section{2014}

\subsection{October}

Read \textcite{shon_how_2012} to help organise step up in quantity of
reading required. Read around possible research topics proposed by Nick

\begin{itemize}
  \item Rumination and autonomous motivation for goal pursuit \parencite{thomsen_people_2011,watkins_constructive_2008}
  \item Experience-sampling of rumination during everyday goal pursuit
    \parencite{moberly_negative_2010,watkins_habitgoal_2014}
  \item Rumination and autobiographical memory accessibility
    \parencite{conway_construction_2000,schoofs_selfdiscrepancy_2013,debeer_operant_2014}
\end{itemize}

Interested in ESM as a method. Met with Mahmood Javaid to chat about
existing phone/web apps suitable for this type of study. Also interested
in potential of counter-conditioning rumination. Identified and proposed
the following as possible gaps to pursue:

\begin{itemize}
  \item \enquote{there is a preponderance of research on RT with
  unconstructive consequences, which needs to be balanced by more
  research into the constructive aspects of RT. In particular,
  more prospective longitudinal studies and experimental studies
  are necessary to investigate the constructive consequences of RT,
  especially in the areas of cognitive processing and post-traumatic
  growth, where most of the evidence is still only cross-sectional}
  \parencite{watkins_constructive_2008}

  \item \enquote{there have been few systematised attempts to account for
  the distinct constructive and unconstructive outcomes of RT (for initial
  suggestions} \parencite{watkins_constructive_2008}

  \item \enquote{Future research will usefully assess RT using alternative
  questionnaires that do not confound RT with the degree of negative
  affect and that can capture other potentially relevant dimensions
  such as the duration, ability to control, and repetitiveness of RT.}
  \parencite{watkins_constructive_2008}

  \item \enquote{Contrary to original predictions, the use of positive
  distractions has not consistently been correlated with lower levels
  of depressive symptoms in correlational studies, although dozens of
  experimental studies show positive distractions relieve depressed mood.}
  \parencite{nolen-hoeksema_rethinking_2008}
\end{itemize}


Proposed the following sketchy design ideas:

\begin{itemize}
  \item Counter-conditioning adaptive/maladaptive rumination
  \begin{itemize}
    \item \enquote{In sum, improvements in mood can temporarily
    reduce rumination by removing a potential cue for the habit.
    However, because of their conservative nature, habits are easily
    reactivated and do not change unless the S-R association itself is
    counter-conditioned.} \parencite{watkins_habitgoal_2014}
    \item Adaptive/maladaptive rumination \parencite{joormann_adadptive_2006}
  \end{itemize}

  \item Abstract/Concrete Construal
  \begin{itemize}
    \item Manipulate Abstract/Concrete thinking \parencite{watkins_constructive_2008}
    \item Outcomes: goal progress/attainment, affect
  \end{itemize}

  \item Brooding vs. reflection
  \begin{itemize}
    \item Manipulate: Brooding/reflection \parencite{moberly_ruminative_2008}
    \item Outcomes: AM specificity \parencite{debeer_associations_2009},
    negative affect \parencite{moberly_ruminative_2008}
    \item RST \parencite{nolen-hoeksema_responses_1991}
  \end{itemize}

  \item Attribution of affect
  \begin{itemize}
    \item Manipulate: attribution of affect \parencite{clore_how_2007}
    \item Outcomes: thoughts, mood
  \end{itemize}
\end{itemize}

Looked at some methodological consequences of ideas which currently interest me

\begin{itemize}
  \item \enquote{such research requires behavioural, physiological, or
    observer-rated outcome measures that reduce the risk of constructive
  outcomes resulting from inaccurate, biased, or defensive self-reports.}
  \parencite{watkins_constructive_2008}

  \item \textcite{csikszentmihalyi_validity_1987}

  \item \enquote{attentional bias (Joorman et al., 2006), sustained pupil
      dilation to negative information (Siegle, Granholm, Ingram, \&
      Matt, 2001; Siegle, Steinhauer, Carter, Ramel, \& Thase, 2003),
      or sustained event-related fMRI amygdala activity in response to
      emotional words (Siegle, Steinhauer, Thase, Stenger, \& Carter, 2002)}
    \parencite{watkins_constructive_2008}
\end{itemize}

\subsection{November}

\begin{itemize}
  \item Chose \textcite{tester-jones_role_2014} as thought paper as a
  way into rumination literature and arranged meeting with Michelle to
  discuss her experiment 4.

  \item \textcite{roberts_cueing_2013} also looks useful as a rumination induction procedure

  \item Losing faith in my prediction that repetitive thought can
  (easily) be moved from positive to negative objects.  Without changing
  their repetitive nature, can negative repetitive thought processes be
  'redirected' towards positive objects/goals? If so, does this elevate
  (trait) positive mood? \parencite{watkins_constructive_2008}

  \item Need to be careful to distinguish rumination from worry
  \parencite{watkins_comparisons_2005,nolen-hoeksema_rethinking_2008}
\end{itemize}

\subsection{December}

Have focused on rumination as a topic. Early reading indicating a
\textit{lot} (RST, S-REF, goal progress theory) of partially overlapping
definitions and corresponding self-report measures \parencite[][p.
206]{papageorgiou_depressive_2004}. \textcite{watkins_habitgoal_2014}
a promising source of questions linking (counter) conditioning to both
RST and control theory. Also pointers to other ideas that might be tested
experimentally e.g. counter-conditioning abstract thinking in negative
rumination contexts.

Read \textcite{watkins_habitgoal_2014}, \textcite[][Chapters 1,
10 (WBSI) and 9, which is a fairly brief summary and comparison
of the main theories in chapters 6-8]{martin_ruminative_1996},
\textcite[][]{smith_roadmap_2009}

\section{2015}

\subsection{January}

\begin{itemize}
  \item Scheduled a weekly meeting with Nick to add some structure and momentum!

  \item Nick kindly gave me feedback on my writing based on an essay I
  gave him to read.

  \item Explaining my understanding of the literature, my rationale and
  the gap I'm addressing to Nick helped clarify it for both of us and
  generated an additional, novel design element. I should have run this
  process sooner and repeatedly, but I feel like I have enough to write
  a proposal grounded in theory now.

  \item Have written draft proposal entitled ``Can attentional training
    reduce rumination?''. A bit behind on reading required for 6/3/2015
    deadline. Need to read the CBM and habitual behaviour literature
    with a view to identifying candidate tasks that might extinguish
    rumination. Worked out that I can use Zotero tags to communicate
    what I've read and am currently reading.

  \item Readings: \parencite{koster_understanding_2011},
    \parencite{almeida_cognitive_2014} and a couple of papers on changing habits.

  \item Looked through slides from case-series workshop given at Exeter
  by Stephen Morley at Exeter. A bit opaque without being there.

  \item Nowhere near a design which makes email regarding first round of project
  funding somewhat moot.

  \item Attended MDC risk training session.
\end{itemize}

\subsection{February}

\begin{itemize}
  \item Working on 2nd proposal draft.
  \item Analysis plan gradually becoming clear having found potential
  tools to run randomisation analysis
    \parencite{bulte_r_2008}, meta-analysis and effect sizes
    \parencite{bulte_singlecase_2013} and produce graphs
    \parencite{bulte_when_2012}.

  \item Basing procedure on \textcite{macleod_cognitive_2012}.
  \item Trying to find a suitable rumination measure/induction combination.
  \item Preparing for presentation of research (proposal).
  \item Read meta-analysis critical of CBM \parencite{cristea_efficacy_2015}.
  \item University virtual machine options for hosting experiment too expensive,
    so planning to host externally. Wondering whether I can get permission to
    reuse the (Internet-delivered) dot-probe from \textcite{see_reduction_2009}.
\end{itemize}

\subsection{March}

\begin{itemize}
  \item Submitted proposal. Need to adjust proposed organisation of
  ideographic and negative word stimuli to effectively detect changes
  in each.

  \item Trying to determine number of participants needed.
    Stephen Morley advised that choosing appropriate N cases for
    meta-analysis is largely heuristic i.e. 4-6, and more = better.
    \textcite{onghena_customization_2005} very useful for deciding on
    individual case-series parameters (measurement times, transitions)
    and N cases for meta-analysis.

  \item Some uncertainty over whether participants will be sufficiently
    motivated to complete the necessary number of session. Have finalised
    that \pounds20 per participant seems appropriate so emailing my case
    requesting \pounds200 for the project.

  \item Felicity pointed me at OpenSesame for dot-probe development in a form
    that could be run by participants at home. Evaluating this against Mahmood's
    recommendation of PsychToolbox or JavaScript.

  \item Trying to find a state rumination measure.
    
  \item Starting work on method. MRC psycho-linguistics DB missing
  valence and arousal data so looking for other data sources for words
  e.g. \textcite{citron_how_2014}. Need to do some C programming to
  get MRC tool suite working! Subsequently decided it was easier to get
  words into a (MySQL) database. More programming! Looks like 1250ms is
  an appropriate stimulus duration. Planning procedure for generating
  ideographic words.  Writing 'sources of support' information sheet
  for participants.

  \item Started Lime Survey learning curve by creating online measurement
    instruments.

  \item Research presentation imminent!
\end{itemize}

\subsection{April}

\begin{itemize}
  \item Various drafts of stimulus schedules to balance stimuli available,
    total number of exposures and organisation so that bias is not towards
    specific stimuli \parencite{see_reduction_2009}. 

  \item Trying to avoid competition for participants with similar, potentially concurrent
    studies.

  \item Had a response from Patrick Onghena regarding p sensitivity I can expect
    in meta-analysis. Not yet resolved.

  \item Materials: Nick resolved my struggle finding STAI and PANAS
  instruments and showed me how to calculate RRS and PSWQ screening
  cutoffs. Started writing preparatory session protocol. Review of
  materials highlighted need for distraction procedure at the end of the
  preparatory session. Took advice from Miriam on using Lime Survey to
  administer the same survey multiple times to a participant. Decided
  to use tokens which meant more Lime Survey learning curve.

  \item Recruitment: Wrote draft of SONA recruitment advert.

  \item Ethics: Pre-submitted application and worked through Nick's feedback.
\end{itemize}

\subsection{May}

\begin{itemize}
  \item Materials: Partial debrief in case procedure fails. Use GAD-7
  instead of STAI. Add sad, depressed, anxious and worried items to
  PANAS. Do trait measures (RRS,PSWQ) before state measures (GAD-7,PHQ-9).
  Starting to balance stimuli properties (frequency, length etc.). Had
  to find an additional dataset for creating I-word pairs as MRC has no
  valence norms. Started task programming in OpenSesame.

  \item Pilot: PhD student piloted screening and preparatory session
  in exchange for doing their study. Nick piloted screening, I-word
  generation and dot-probe task. Need to handle generated proper
  nouns. I-word valence ratings a bit tricky.

  \item Assessment: Concluded from my inability to answer Fraser's
  question during presentation that there's more evidence ${that}$ ABM
  works than ${how}$ it works! Happy with poster submitted.

  \item No word on budget approval.

  \item Ethics: Submitted application. Re-read current MDC risk protocol
    procedure.
\end{itemize}

\subsection{June}
\begin{itemize}
  \item Reduced I-words to 6 from 8 to ease generation and simplify
  stimulus allocation.

  \item Admin: \pounds200 budget approved and collected. Getting up to
  speed with room booking arrangements in advance of pilot. Made arrangements
  for times during study when I and Nick will be away. Addressed minor issue in
  conditional ethical approval.

  \item Dot-probe: Removed timeout and incorrect response
  feedback. Corrected word sizes. Various fiddly operating-system software
  distribution issues. Considering loaning mice to people as Mac users
  will only have a trackpad, but rejected idea as too complicated and
  reverted to keyboard responses which is common in literature. Borrowed
  Mac from John Staplehurst for testing.

  \item Further tweaks to participant documents and email
  templates. Rewording instructions to balance getting genuine current
  concerns without participants having to reveal sensitive information.
  Participants have to complete 25 sessions (minimum data needed for
  analysis) for \pounds10 then get \pounds1 for each additional session.

  \item After some discussion, decided that I ${would}$ use PHQ-9 rather
  than PHQ-8. I feel responsible for participants so want to leave
  suicidal ideation question in and be prepared to run risk protocol
  and have clinical cover in place when needed. Kim Wright can provide
  risk cover but advised adding BDI suicidal ideation question to Lime
  Survey as risk protocol trigger.

  \item Piloting: finding more versions of OSX which can't run OpenSesame,
  problems choosing I-word pairs, dot-probe data file upload issues
  (Lime Survey). Nick tested adjustment procedure to deal with illegal I-words,
  which adds 5-10 minutes to preparatory session.

  \item Created Windows and Mac software installation videos and uploaded to
  YouTube.

  \item Recruitment: Activated SONA advert. Looking for University mailouts, and
    departmental lists which are OK for an advert. Some suspended over the summer.

\end{itemize}

\subsection{July}

\begin{itemize}
  \item Patrick Onghena didn't notice my supplementary question in April!
  Meta-analysis p sensitivity looks OK.

  \item Co-coordinating people available to provide clinical cover
  (especially over the summer) to coincide with participant preparatory
  sessions proving time consuming, labour intensive and frustrating.

  \item Recruitment a bit slow so have submitted updated ethics request
  to increase PHQ-8 to <= 11. Nick (The Sorcerer) wrote most of the
  magic words to make this happen while the apprentice watched.

  \item Procedure: Running smoothly with first participants. Nick
  double checking my I-word pair matching. Having to deal with one or
  two tech support issues but they're getting resolved. First Mac users
  are getting the task running fine (worth the extensive testing!). A
  few problems with people attaching dot-probe data files but having them
  email these to me is fine as a workaround. Skipping sessions is common
  (reasons: life, Internet etc.) as expected but mostly the odd day here
  or there. One participant who was missing lots of consecutive sessions
  eventually dropped out. One participant withdrew after preparatory
  session but before starting daily sessions (sent support info and
  partial debrief). Managed to keep things going whilst away at PsyPAG
  but a bit draining sending out correct session reminders to multiple
  participants last thing at night. Have created a spreadsheet to keep
  track of where everyone is in their schedule. Now have ${\geq}$ 10 Ps
  that I proposed but keeping people in (and more in reserve) in lieu
  of dropout.  Creating canned emails for milestones such as transition
  to ABM training.

  \item Admin: Submitted mitigation request as it looks like final
  starters won't end until very close to submission deadline.

\end{itemize}

\subsection{August}

\begin{itemize}
  \item Analysis: Bulk of the code (perl/R) written to tidy and analyse
  data. Correlation analyses to see which PANAS supplementary items are
  worth keeping. Cronbach's tests for daily measures. Concluded that
  multiple baseline analysis wasn't possible because participants will
  complete different numbers of sessions. Clarification on appropriate
  SD for Cohen's d calculation. Descriptive stats.

  \item Commit history: https://github.com/earcanal/dotprobe/commits/master

  \item Reading: Need to do quite a lot in preparation
  for introduction and discussion. Did a second literature
  search and two rounds of pruning after running it past Nick.
  Current concerns \parencite{klinger_motivation_2011}, worry
  \parencite{borkovec_preliminary_1983}.

  \item Dissertation: Good decision to use LaTeX. A bit of a (re)learning
  curve but it's the easiest way of repeatedly running analyses and
  pulling output directly into the report. Nick looking over early drafts.

  \item Participants: Definitely need more motivation in phase B when
  the sessions get longer. One P opted to complete a session every 3 days
  (my suggestion) rather than dropout. Another decided to stop at session
  25. Ps now completing the study so getting into the swing of running
  the final session and being prepared for risk protocol triggers with
  remote Ps. Payment process working fine. Paid one P who couldn't quite
  make it to 25 sessions even thought I can't use their data.

  \item Admin: Mitigation approved.
\end{itemize}

\subsection{September}

\begin{itemize}
  \item Analysis: Hacking SCVA to standardise y-axes and get multiple
  plots on the same graph. Generate tables for inferential stats. Final
  decision that randomisation tests should be 2-tailed. As last data is coming
  in the analyse/bug fix spiral is getting tighter.

  \item Reading: Re-read of \textcite{shon_how_2012} very worthwhile. Have
  adopted this process using Zotero features which is making the large
  reading task directed and enjoyable. Not really working from the papers
  identified in literature search. As I read and look over my results,
  I'm finding that I'm following up one or two papers around points that
  are emerging as being important. My conclusions are shifting a little
  the more I read and write. I think I'm able to read faster and more
  tactically now, partly out of necessity as I'm still finding things
  I wish I'd read much earlier.  Re-reading key papers also important
  and useful. There are many things I missed or mis-understood on first
  reading. With hindsight I realise a little more reading and some more
  literature searches early on would have resulted in a much better
  research question. I read \textcite{cristea_efficacy_2015} six months
  before the debate between CBT and CBM landed properly!

  \item Dissertation: Good decision to use LaTeX. A bit of a (re)learning
  curve but it's the easiest way of repeatedly running analyses and
  pulling output directly into the report. Nick's comments on  early
  and final draft useful for tidying up everything as far as the
  discussion. Structuring discussion using Nicks advice and themes I
  thought emerged seemed fairly successful.  Numerous reading/writing
  cycles required to complete it. Marking up a written copy was definitely
  the most effective way of touching up the final draft.
\end{itemize}

\printbibliography
\end{document}
